\label{ch:introduction}
The present dissertation seeks to report about the work that was undertaken to
predict the sensitivity of a future neutrino experiment to the underlying
oscillation parameters. 
The production of such results was achieved through the implementation of the
quantum mechanical description of neutrino oscillations in a self-contained C++
framework, combined with preliminary models of said experiment.

For this reason, we start the next chapter by introducing the neutrino and the
concept of neutrino oscillations with its associated formalism. From the
starting point that neutrino flavour eigenstates are different from neutrino
mass eigenstates, we derive the probability of a neutrino changing flavour
after travelling through empty space. 

In our model of a neutrino experiment, we use the assumption that three
neutrino flavours exist and that a neutrino produced in a particular flavour
will oscillate to the other two and back. Hence we introduce the mixing matrix
for three flavours, known as the PMNS matrix, and emphasize the presence in this
matrix of an imaginary phase $\me^{\mi\delta_{CP}}$ which for $\delta_{CP}
\neq 0$ would generate CP violation in the lepton sector.

Because the neutrino experiments under consideration are long-baseline
underground experiments, it is necessary to treat the flavour oscillations not
in vacuum but in matter. We show a derivation of the corrections that must be
applied in the case where only two neutrinos oscillate and discuss how one can
extend this framework to the three neutrino case.

In order to provide motivation for our work, we examine the current state of
neutrino oscillations, namely which parameters are known and which parameters
require different experiments or more data in order to be determined. We
introduce the mass hierarchy problem, which, along with the CP-violating phase
$\delta_{CP}$, will be our two oscillation parameters of interest in what
follows.


While chapter~\ref{ch:osc} covers the fundamental principles at work, some elements of
statistics also need to be discussed. Evaluating the sensitivity of an
experiment requires the formal definition of a relevant test statistic, which
then has to be evaluated from our model of the experiment. In order to make the
results meaningful, we must also argue about the interpretation of such a test
statistic in the context of an experiment which has yet to be performed.
The statistics are discussed in the first part of chapter~\ref{ch:methods}.

We then present the oscillation probabilities resulting from our
implementation of the formalism of chapter~\ref{ch:osc} and try to outline some
relevant features. 
The whole purpose of such a model is to apply it to a real neutrino experiment
and study the potential outcomes and discoveries that it could bring.
We chose the Deep Underground Neutrino Experiment (DUNE) as our main
application. Because it is already a very mature project, a multitude of
extremely detailed resources are available, such as the four-volume conceptual
design report\cite{cdr-all} (CDR). Using information and results from the CDR
lets us cirumvent the in-depth modelling of a neutrino beam and detector, and
focus on the oscillations instead. This is discussed in detail in the second
half of chapter~\ref{ch:methods}.

Finally, we combine all the information of chapter~\ref{ch:methods} into our
predictions of the sensitivity of DUNE to CP violation and to the neutrino mass
hierarchy, the two most important unknown oscillation parameters.
As a short extra application, we produce similar results for the shorter-baseline
Hyper-Kamiokande (Hyper-K) experiment. We examine the differences between DUNE
and Hyper-K, and discuss the potential complementarities between the two
future experiments.

