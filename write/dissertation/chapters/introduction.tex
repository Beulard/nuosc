% This chapter gives reader the background knowledge they will need in order to understand the following chapters. This should not include descriptions of the work you have done, but it should provide the motivation for why you did it.  Some reports will have two chapters of background material (e.g. one on general background from books, and one providing a review of journal articles you have read).
\label{ch:introduction}
This dissertation is a report of the steps that were taken in order to predict the
sensitivity of a future neutrino experiment to the neutrino mass hierarchy and
to CP violation.
The work that was done heavily relies on the quantum mechanical description of
neutrino oscillations and the implementation thereof in a self-contained C++
framework.

For this reason, we start chapter~\ref{ch:osc} by introducing the neutrino and the
concept of neutrino oscillations with its associated formalism. From the
starting point that neutrino flavour eigenstates are different from neutrino
mass eigenstates, we derive the probability of a neutrino changing flavour
after travelling through empty space. 

In our model of a neutrino experiment, we use the assumption that three
neutrino flavours exist and that a neutrino produced in a particular flavour
can oscillate to the other two and back. Hence we introduce the mixing matrix
for three flavours, known as the PMNS matrix, and discuss the presence in this
matrix of the imaginary phase $\me^{\mi\delta_{CP}}$ which could be responsible for CP
violation in the lepton sector.

Because the neutrino experiments under consideration are long-baseline
underground experiments, it is necessary to treat the flavour oscillations not
in vacuum but in matter. We show a derivation of the corrections that must be
applied in the case where only two neutrinos oscillate and discuss how one can
extend to the three neutrino case.

In order to provide motivation for our work, we discuss the current state of
neutrino oscillations, namely which parameters are known and which parameters
require higher precision experiments in order to be measured.


While chapter~\ref{ch:osc} covers the fundamental principles at work, some elements of
statistics also need to be discussed. Evaluating the sensitivity of an
experiment requires the formal definition of a relevant test statistic, which
then has to be evaluated from our model of the experiment. In order to make the
results meaningful, we must also discuss the interpretation of such a test
statistic in the context of an experiment which has yet to be performed.
The statistics are discussed in the first part of chapter~\ref{ch:methods}.

We then present the oscillation probabilities resulting from our
implementation of the formalism of chapter~\ref{ch:osc} and try to outline some
relevant features. 

The whole purpose of such a model is to apply it to a real neutrino experiment
and study the potential outcomes and discoveries that it could bring.
We chose the Deep Underground Neutrino Experiment (DUNE) as our main
application. Because it is already a very mature project, a multitude of
extremely detailed resources are available, such as the four-volume conceptual
design report\cite{cdr-all} (CDR). Using information and results from the CDR
lets us cirumvent the in-depth modelling of a neutrino beam and detector, and
focus on the oscillations instead. This is discussed in detail in the second
half of chapter~\ref{ch:methods}.

Finally, we combine all the information of chapter~\ref{ch:methods} into our
predictions of the sensitivity of DUNE to CP violation and to the neutrino mass
hierarchy.
As a short extra example, we produce similar results for the shorter-baseline
Hyper-Kamiokande (Hyper-K) experiment and use that as a reference to
compare to DUNE.


%This chapter introduces the report and tells the reader what they should expect
%to see in the next chapters. We should introduce the contents of the
%following two chapters in a qualitative way and maybe give some context and
%motivation for the research.
