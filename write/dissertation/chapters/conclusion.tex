\label{ch:conclusion}
\section{Summary}
It is clear that the next generation of experiments
is being designed specifically to solve the remaining issues in the theory of
neutrino oscillations. Using a simple model of neutrino oscillations and
preliminary results from the designing groups, we are able to show
quantitatively how sensitive we expect DUNE and Hyper-K to be to two of the
as-yet-unknown oscillation parameters, namely the neutrino mass hierarchy and
the CP-violating phase $\delta_{CP}$.

From figure~\ref{fig:sens_mh}, we predict that DUNE, with an exposure of 150
kt$\cdot$MW$\cdot$year, will reject one of the hierarchies with a significance
greater than $5\sigma$ under any combination of mass hierarchy and
$\delta_{CP}$. DUNE's sensitivity to CP violation (fig.~\ref{fig:sens_cp}) is
fairly low, exceeding a $5\sigma$ significance for only about 10\% of possible
$\delta_{CP}$ values, for either hierarchy.

We predict that Hyper-K, after a 10-year exposure to the J-PARC beam, is
much less sensitive to the mass hierarchy than DUNE despite recording more
events. Its maximum mass hierarchy rejection significance is around $3\sigma$
(fig.~\ref{fig:hk_sens_mh}). This poor sensitivity is due to the relatively
short baseline (295 km). Because matter effects are weak at this baseline, the
differences in the oscillation probability due to the mass hierarchy are small.
The sensitivity of Hyper-K to CP violation (fig.~\ref{fig:hk_sens_cp}) appears
exceedingly high, being better than $5\sigma$ for 75\% of the possible
$\delta_{CP}$ values under either hierarchy. Such a high sensitivity can be
seen as the result of suitable design choices for the beam and detector,
which maximizes the expected event rate over the given exposure as well as the
accuracy of the energy reconstruction process.


Evidently, the most beneficial outcome for neutrino physics would result from
combining the two experiments, with DUNE rejecting one of the hierarchies and
Hyper-K confirming or disproving CP violation. In reality, many more neutrino
experiments are currently running and being designed, and the bigger picture is
full of interesting details that are beyond the scope of this report, such as
parameter degeneracies, the existence of sterile neutrinos or non-standard
neutrino interations (see~\cite{raut}).

Thus what we present here is merely a simplistic and incomplete picture, but it is
hopefully a worthwhile attempt at providing insight into the future of neutrino
oscillation experiments.


\section{Shortcomings and future work}
There are many ways in which this project is incomplete, some of which stand
out more than others. 
One of them is that our model only considers the appearance events resulting
from the neutrino mode of the respective accelerator. Both the LBNF and J-PARC
beams are designed to be able to run in neutrino or anti-neutrino mode,
i.e.~it is possible to reverse the focusing magnetic field such that
anti-neutrinos are predominantly present in the beam. Both DUNE and Hyper-K are
planned to collect data under both polarities.  In addition to increasing the
overall sensitivity (since more data is collected), this has the effect of
breaking the $\theta_{23}$ octant degeneracy, as discussed in section~\ref{sec:state}.

Incidentally, there is also a parameter degeneracy in the combination of mass
hierarchy and $\theta_{23}$ octant that cannot be resolved unless measurements
in anti-neutrino mode are made as well\cite{raut}. If we denote by LO the case
where $\theta_{23}$ lies in the lower octant and by HO the case where it lies
in the higher octant, the degeneracy occurs between the combinations (NH, LO)
and (IH, HO). That is, these two combinations of parameters result in the same
oscillation probability.
In theory, this would prevent DUNE from determining the true mass hierarchy if
it only made measurements in neutrino mode, as long as the octant of
$\theta_{23}$ was undetermined. To avoid this problem in our model, and for
simplicity, $\theta_{23}$ was assumed to lie in the lower octant.
A better model would consider both polarity modes and add the $\theta_{23}$
parameter to the analysis, resulting in an extra dimension in parameter space
and a greater and more accurate insight into the
potential outcomes of the experiment.

Secondly, as pointed out in section~\ref{sec:sens_cpv}, we do not explicitely
evaluate the distribution of the CP violation test statistic
$\Delta\chi^2_{CPV}$, which impedes our ability to interpret CP violation
sensitivity results. A more thorough study would include explicit determination
of the distributions of both the mass hierarchy test statistic $\Delta\chi^2$
of equation~\ref{eq:dc2} as well as $\Delta\chi^2_{CPV}$, through Monte Carlo
simulations of the experiment.

Thirdly, our results are given only for a fixed value of the exposure of the detector to
the beam. A straightforward extension to the code would allow us to determine
the sensitivity \emph{as a function of exposure}. From this information, a
value of the ``critical exposure'' could be extracted.  We could define
this value, for example, as the exposure for which the mass hierarchy
rejection significance is greater than $5\sigma$ for all values of
$\delta_{CP}$. Coincidentally, for DUNE, our results seem to indicate that this
critical exposure is quite close to 150 kt$\cdot$MW$\cdot$year (see
figure~\ref{fig:sens_mh}).

In section~\ref{sec:hyperk}, we quickly mentioned the construction of a second
Hyper-K detector in Korea. Again, with minor additions to the code, the
increase in sensitivity of the Hyper-K experiment could have been evaluated, 
leading to more relevant results.

Finally, and although it would require significantly more work to make
the code flexible enough, the model could be expanded to atmospheric neutrinos,
in order to more realistically model Hyper-K, and to allow the incorporation of
experiments such as IceCube collaboration's PINGU, ICAL, and others.


%%All reports should include a conclusions chapter. This can be short. This can collate conclusions of previous chapters (these can be direct copies, rather than reworded, but ask your supervisor to check that this is OK in your specific context).
%%
%%Some conclusions chapters will include a section describing future work. 
%%
%%Take the opportunity to reflect on your experience. If you were able to go back in time 9 months, what advice would you give yourself?
