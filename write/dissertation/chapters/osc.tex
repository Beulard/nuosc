\label{ch:osc}
We will now give an overview of the theoretical concepts that are used
throughout this report. We will present a review of neutrino flavour
oscillations in a vacuum, for the commonly considered cases of two and three
neutrino flavours. We will derive the oscillation probability for
neutrinos in matter with a constant density, in the simpler case of two
neutrinos. In the last section, the current state of neutrino oscillations will
be reviewed in order to provide context and motivation for the next chapter.


\section{History of the neutrino}
The neutrino particle was first postulated by Wolfgang Pauli in 1930 to explain
the continuous energy distribution of the electron in beta decays, in his
famous letter to the Physics Institute of Zürich\cite{pauli}. 

Its existence was confirmed in 1956 by Cowan and Reines\cite{cowan} through the
observation of anti-neutrinos from a nuclear reactor being captured by protons
in a water tank.
A number of subsequent experiments were designed to map out the properties of
the neutrino and its interactions\cite{zuber}. As of today, the neutrino is
still not fully understood, and neutrino physics has become one of the leading
branches of experimental particle physics along with collider physics. 

The existence of distinct flavours of neutrinos was first investigated by
Bruno Pontecorvo\cite{pontecorvo} in 1959. He proposed that a weak interaction
involving charged leptons ($e$, $\mu$) discriminates between neutrino flavours,
hence a process such as $\nu_\mu + n \rightarrow e^- + p$ is forbidden.
This was confirmed in 1962 by Danby et al.\cite{danby} at the Brookhaven
National Laboratory, where they showed that electron-neutrinos and
muon-neutrinos were in fact distinct particles.
This was also a first hint at the possibility of neutrino flavour oscillations,
which Pontecorvo\cite{pontecorvo-osc} started theorizing in 1967.
In 2000, the DONUT experiment\cite{donut} reported the observation of
tau-neutrino interactions, hence further corroborating the theory of the
three neutrino flavours.

As will be discussed later, the fact that a neutrino can change its flavour by
propagating through empty space is direct evidence that neutrinos are massive
particles, unlike what is initially described by the Standard Model.
The 2015 Nobel Prize was awarded to the Super-K and Sudbury Neutrino
Observatory (SNO) collaborations for the discovery of neutrino oscillations in
1999 and 2001\footnote{In fact, SNO observed oscillations in matter, known as
the MSW effect\cite{smirnov}, while Super-K observed oscillations in
vacuum.}, respectively.

Although neutrino oscillations have been observed, some of the physical parameters that
govern this process remain difficult to probe because of the elusiveness of
neutrinos: being only involved in weak processes, they require extremely
voluminous detectors in order to get the slightest signal.
The Deep Underground Neutrino Experiment (DUNE) in the United States and the
upgrade to Super-Kamiokande, Hyper-Kamiokande in Japan, are examples of
future experiments that aim to determine these parameters.

In the Standard Model, there are twelve elementary fermions: six quarks and six
leptons (plus their respective antiparticles). The leptons consist of three
charged leptons $(e, \mu, \tau)$ and three corresponding charge-less neutrinos
$(\nu_e, \nu_\mu, \nu_\tau)$. The electron neutrino $\nu_e$, for example, is
the neutrino which can interact weakly with an electron and a
$W$ boson~(Fig.~\ref{fig:weak_vertex}). 

\begin{wrapfigure}{r}[1.5cm]{0.2\textwidth}
	\includegraphics{images/weak_vertex.pdf}
	\captionsetup{justification=centering}
	\caption{The weak vertex for an electron neutrino.}
	\label{fig:weak_vertex}
\end{wrapfigure}

Neutrinos, being leptons (colorless) and carrying zero charge, interact only via the weak
interaction, which makes them impossible to observe directly. Their existence
and their properties can only be inferred from their interactions with other
particles.
Experimentally, only left-handed neutrinos and right-handed
anti-neutrinos are observed. In the Standard Model, this curious property
prevents the neutrinos from acquiring a mass through interactions with the
Higgs field in the same way that charged leptons do\cite{schwichtenberg}. Hence
the Standard Model on its own cannot accurately describe neutrinos, and
extensions such as the seesaw mechanism or extra dimensions are needed. This
discussion lies beyond the scope of this report since we are only concerned
with the phenomenological description of neutrino oscillations, which is
achieved via straightforward quantum mechanics in the next section.

\section{Neutrino oscillations}
We will now introduce the mathematical tools that we use to describe neutrino
oscillations. 

It is now known experimentally that 
the neutrino flavours ($\nu_e, \nu_\mu, \nu_\tau$) mix when propagating through
space\cite{superk}. This process is structurally similar to the quark generation
mixing between down, strange and bottom quarks and is direct evidence that for
neutrinos and quarks, the weak eigenstates\footnote{Weak eigenstate: state
which interacts with the W, Z bosons.} are not in one-to-one correspondence
with the mass eigenstates\footnote{Mass eigenstate: state which has definite
mass and propagates through space as a wave.}.

In the Standard Model, the W bosons mediate the interactions between up-type and
down-type quarks, and between charged leptons and neutrinos. 
When describing these particles, it is useful to express them either in the
weak eigenbasis for interactions or in the mass eigenbasis for
free-propagation.
For down-type quarks and neutrinos, these bases are not identical: their
relationship is conventionally represented by the Cabibbo-Kobayashi-Maskawa
(CKM) matrix and the Pontecorvo-Maki-Nakagawa-Sakata (PMNS) matrix,
respectively.  This allows down-type quarks to interact with up-type quarks of
a different generation, or, as we shall see, it allows neutrinos to change their
flavour composition as they propagate in empty space.


\subsection{Neutrino oscillation formalism} 
The following derivations are heavily inspired by chapter 8 of~\cite{zuber}.
Let us consider a general system where there exist $n$ neutrino flavours. 
We mentioned the two eigenbases of interest: the weak (flavour)
eigenstates $\ket{\nu_{\alpha}}$ where $\alpha$ is a lepton flavour,
$\alpha=l_1, l_2, ..., l_n$, and the mass eigenstates $\ket{\nu_i}$, $i=1, 2,
..., n$. 
Flavour eigenstates interact with matter through the weak interaction, and
hence are the states we observe in Nature. Eigenstates are orthogonal within
each basis:
$$
\bra{\nu_\alpha}\ket{\nu_\beta} = \delta_{\alpha \beta}, \quad
	\bra{\nu_i}\ket{\nu_j} = \delta_{ij}  \quad
$$
and the two bases are related by a $n\times n$ unitary matrix $U$:
\begin{align}
	&\ket{\nu_\alpha} = \sum_i U_{\alpha i} \ket{\nu_i}\label{eq:bases1}\\
	&\ket{\nu_i} = \sum_\gamma (U^\dagger)_{i \gamma} \ket{\nu_\gamma} = \sum_\gamma
			U^*_{\gamma i} \ket{\nu_\gamma}\label{eq:bases2}\\
	&U^\dagger U = 1, \nonumber
\end{align}
where greek letters denote indices in the flavour basis and latin letters
denote indices in the mass basis.

The mass eigenstates are solutions of the free Hamiltonian in a vacuum, hence
they are stationary states and evolve in time as $\ket{\nu_i(x,
t)} = \me^{-\mi E_i t} \ket{\nu_i(x, 0)}$, where $\ket{\nu_i(x, 0)} = \me^{\mi p
x} \ket{\nu_i}$ for a plane wave neutrino produced at $x=0$.
Hence a neutrino produced as a flavour eigenstate $\alpha$ would have the
following time dependence:
\begin{align*}
\ket{\nu_\alpha(x, t)} &= \sum_i U_{\alpha i} \ket{\nu_i(x, t)} \\
		&= \sum_i U_{\alpha i} \me^{-\mi E_i t} \me^{\mi p x} \ket{\nu_i}\\
		&= \sum_{i, \gamma} U_{\alpha i} U^*_{\gamma i} \me^{\mi p x}
				\me^{-\mi E_i t} \ket{\nu_\gamma}\quad\text{(using eq.~\ref{eq:bases2})}.
\end{align*}
Note that we are working in natural units where $c=\hbar=1$ for clarity. When
implementing this formalism in code, the units must be restored using
the usual conversion factors in order to get accurate numerical results.
The transition amplitude to a flavour $\beta$ is thus 
\begin{align}
	A_{\alpha \rightarrow \beta}(x, t) = \bra{\nu_\beta}\ket{\nu_\alpha(x, t)} \nonumber
			= \sum_i U^*_{\beta i} U_{\alpha i} \me^{\mi p x}
			\me^{-\mi E_i t}\quad\text{(using eq.~\ref{eq:bases1})}.
\end{align}
By assuming that all neutrinos have small masses, and thus are highly
relativistic, we are able to perform a binomial approximation on their
energy: 
\begin{equation} E_i = \sqrt{m_i^2 + p^2} \simeq p + \frac{m_i^2}{2 p} \simeq E +
\frac{m_i^2}{2E},\label{eq:binomial}\end{equation}
where we call $E \approx p$ the neutrino energy at the source\footnote{As
pointed out by Cohen, Glashow and Ligeti\cite{cohen}, we are taking a shortcut
here by assuming that the weak eigenstate comes out as a linear superposition
of mass eigenstates with equal energies. However our description leads to
identical oscillation formulae as a more physically rigorous one where particle
entanglement after the interaction is taken into account. We choose to keep
this derivation and skip the details for the sake of simplicity.}. We call $L$
the distance from the neutrino source to the detector, known as the baseline.
Under our relativistic assumption, we have $L \approx t$. Going back to the
transition amplitude,
\begin{align*}
	A_{\alpha \rightarrow \beta}(L, E) &= \sum_i U^*_{\beta i} U_{\alpha i}
	\exp(\mi E L - \mi(E + \frac{m_i^2}{2 E}) L)\\
	&= \sum_i U^*_{\beta i} U_{\alpha i} \exp(-\mi \frac{m_i^2 L}{2E}),
\end{align*}
The transition probability is then
\begin{align}
	\label{eq:nuprob}
	P_{\alpha \rightarrow \beta}(L, E) &= |A_{\alpha \rightarrow \beta}(L, E)|^2 \nonumber\\
	&= \sum_i \sum_j U_{\alpha i} U^*_{\beta i} U^*_{\alpha j} U_{\beta j}
	\exp(-\mi \frac{\Delta m_{ij}^2}{2} \frac{L}{E}),
\end{align}
where $\Delta m^2_{ij} = m^2_i - m^2_j$ is the mass-squared difference between
two mass eigenstates.
The analysis is similar when considering anti-neutrinos except that $U$ must be
replaced by $U^*$ everywhere and vice-versa.

The formula we have derived here is the general probability of oscillation from a
flavour eigenstate $\ket{\nu_\alpha}$ to another flavour eigenstate
$\ket{\nu_\beta}$. This effectively describes the probability that after
travelling a distance $L$, the neutrino wavefunction will collapse into a
$\beta$ flavoured neutrino if it interacts in our detector.
In the following sections, we apply this formalism to two-neutrino and
three-neutrino systems. 



\subsection{Two-neutrino oscillations}\label{sec:twonu}
The case for two neutrino eigenstates is the simplest to consider. The
unitary nature of the mixing matrix $U$ and the arbitrarity of the complex phases
of quantum states reduces the number of observable mixing parameters to
one\cite{langacker}.
An enlightening way to write the mixing matrix is\cite{zuber} 
\begin{equation}
	\begin{bmatrix} \nu_\mu \\ \nu_\tau \end{bmatrix} =
	\begin{bmatrix} \cos\theta & \sin\theta \\
								 -\sin\theta & \cos\theta \end{bmatrix}
	\begin{bmatrix} \nu_1 \\ \nu_2 \end{bmatrix},
		\label{eq:twonumixing}
\end{equation}
where $\theta$ is called the mixing angle.
We see here that the two eigenbases are analogous to two sets of Cartesian axes
in two dimensions, one being rotated by an angle $\theta$ with respect to the
other.

We use the $\mu$ and $\tau$ flavours here in reference to the
atmospheric neutrino experiment at Super-Kamiokande\cite{superk} (Super-K): for
neutrinos produced by cosmic rays in the atmosphere, it is a good approximation
to consider a two-neutrino oscillation between the $\mu$ and $\tau$ flavours,
and to neglect the $\nu_\mu \rightarrow \nu_e$ channel. 
This simple mixing matrix can be substituted in eq.~\ref{eq:nuprob} to obtain a
more concrete oscillation probability:
\begin{align}
	P_{\nu_\mu \rightarrow \nu_\tau}(L, E) &= U_{\mu 1}^2 U_{\tau 1}^2 + U_{\mu 2}^2
	U_{\tau 2}^2 \nonumber\\&\quad+ U_{\mu 1} U_{\tau 1} U_{\mu 2} U_{\tau 2} \bigg[\exp(-\mi
	\frac{\Delta m_{12}^2 L}{2 E}) + \exp(-\mi \frac{\Delta m_{21}^2 L}{2
	E})\bigg]\nonumber\\
	&= \cos^2\theta \sin^2\theta\enskip \bigg[2 - 2\cos\frac{\Delta m_{21}^2 L}{2
	E}\bigg]\nonumber\\
	&= \sin^2(2\theta)\sin^2\bigg(\frac{\Delta m^2}{4}
	\frac{L}{E}\bigg),\label{eq:twonu}
\end{align}
where in the first step we have used the fact that $\Delta m_{12}^2 = m_1^2 -
m_2^2=-(m_2^2 - m_1^2) = -\Delta m_{21}^2$ to bring the exponentials into
cosine form.
Note that there are only two mass eigenstates here, hence we define $\Delta m^2
\equiv \Delta m_{21}^2$.

This example is particularly informative, as one can clearly see the role of
each parameter in the oscillation probability. The first term is a constant
which depends on the mixing angle, and must be determined by experiment. For
$\theta = \frac{\pi}{4}$, the mixing is maximal and the probability is allowed
to increase up to 100\%. The second term is manifestly oscillatory. Its
frequency is determined by the mass difference $\Delta m^2$, which may also be
determined by fitting experimental data. Note also that if the mass squared
difference is zero, the oscillation probability is zero. This implies that
oscillations are only possible if at least one of the mass eigenstates has a
non-zero mass.
Evidently, since there are only two flavours, the probability of $\nu_\mu$
``survival'' is
$$P_{\nu_\mu \rightarrow \nu_\mu}(L, E) = 1 - P_{\nu_\mu \rightarrow
\nu_\tau}(L, E).$$



\subsection{Three neutrino oscillations}
When we consider three eigenstates, the number of parameters is increased to
four\cite{langacker}:
three mixing angles $\theta_{12}, \theta_{13}$ and $\theta_{23}$, and an
imaginary phase $\me^{\mi \delta_{CP}}$. This so called CP-violating phase gives rise to
a difference between oscillation probabilities of neutrinos and anti-neutrinos,
hence it allows violation of the charge-parity (CP) symmetry. 
The mixing matrix can be written in a suggestive way as
\begin{equation}\label{eq:PMNS}
U_{\mathrm{PMNS}} = 
\begin{bmatrix} 1 & 0 & 0 \\ 0 & c_{23} & s_{23} \\ 0 & -s_{23} & c_{23} \end{bmatrix}
\begin{bmatrix} c_{13} & 0 & s_{13} \me^{-\mi \delta_{CP}} \\ 0 & 1 & 0 \\
								-s_{13} \me^{\mi \delta_{CP}} & 0 & c_{13} \end{bmatrix} 
\begin{bmatrix} c_{12} & s_{12} & 0 \\ -s_{12} & c_{12} & 0 \\ 0 & 0 & 1 \end{bmatrix}
,\end{equation}
where $c_{ij} = \cos(\theta_{ij})$ and $s_{ij} = \sin(\theta_{ij})$.
This form is analogous to a 3D rotation around three Euler angles, except for
the appearance of $\delta_{CP}$ in the ``13'' rotation.

Unlike in the two-neutrino case, substituting this mixing matrix into
eq.~\ref{eq:nuprob} turns out not to be very insightful. It is important
however to underline all the oscillation parameters that are present here.
Along with the four mixing parameters in the PMNS matrix, we have three mass
squared differences $\Delta m^2_{31},~\Delta m^2_{32}$ and $\Delta m^2_{21}$,
of which two are independent since $\Delta m^2_{32} + \Delta m^2_{21} = \Delta
m^2_{31}$. This leaves six independent
parameters that must be determined by oscillation experiments.
In section~\ref{sec:state}, we will go over the current state of our knowledge
of these parameters, which will provide motivation for the work presented in
chapter~\ref{ch:methods}.


\subsection{Neutrino oscillations in matter}
So far we have only discussed neutrino oscillations in vacuum, where no weak
interactions take place.
In matter, where atoms are present, neutrinos can undergo coherent forward
scattering off of atomic electrons, modifying the oscillation probability over
large enough distances. This is known as the Mikheyev-Smirnov-Wolfenstein (MSW)
effect\cite{wolfenstein, mikheyev-smirnov}, or simply matter effect.
\begin{figure}
	\centering
	\begin{subfigure}{0.3\textwidth}
		\centering
	\feynmandiagram[small, vertical=a to b]{
		ni [particle=\(\nu_e\)] -- [fermion] b -- [fermion] ef [particle=\(e^-\)],
		a -- [boson, edge label=\(W\)] b,
		ei [particle=\(e^-\)] -- [fermion] a -- [fermion] nf [particle=\(\nu_e\)],
		};
		\caption{Electron-neutrino scattering.}
		\label{fig:nu_e-scatter}
	\end{subfigure}
	\begin{subfigure}{0.3\textwidth}
		\centering
		\feynmandiagram[small, horizontal=a to b]{
			ei [particle=\(e^-\)] -- [fermion] a -- [fermion] ni
			[particle=\(\bar{\nu}_e\)],
			a -- [boson, edge label=\(W^-\)] b,
			nf [particle=\(\bar{\nu}_e\)] -- [fermion] b -- [fermion] ef
			[particle=\(e^-\)],
			};
			\caption{Anti-$\nu_e$ scattering.}
			\label{fig:anti-nu_e-scatter}
	\end{subfigure}
	\begin{subfigure}{0.3\textwidth}
		\centering
		\feynmandiagram[small, vertical=a to b]{
			ni [particle=\(\nu_{e,\mu,\tau}\)] -- [fermion] b -- [fermion] nf
			[particle=\(\nu_{e,\mu,\tau}\)],
			a -- [boson, edge label=\(Z\)] b,
			ei [particle=\(e^-\)] -- [fermion] a -- [fermion] ef [particle=\(e^-\)],
		};
		\caption{Scattering through $Z$ boson affects all flavours equally.}
		\label{fig:z-scatter}
	\end{subfigure}
	\caption{The three possible tree-level interactions of neutrinos with atomic
	electrons in matter.}
\label{fig:forward_scatter}
\end{figure}

A neutrino can scatter either by exchanging a $W^\pm$ boson --- known as
charged current (CC) interaction --- or by exchanging a $Z$ boson --- neutral
current (NC) interaction. The neutral current interaction is experienced
equally by all three flavours (fig.~\ref{fig:z-scatter}), hence it does not
have an effect on the relative
propagation phases, and causes no change in the oscillation probability. The
charged current is specific to the neutrino flavour. Since only
electrons are present in matter, only electron neutrinos are subject to CC
interactions~(figs.~\ref{fig:nu_e-scatter},~\ref{fig:anti-nu_e-scatter}). 
Hence electron neutrinos propagating in matter pick up a different phase than
if they were propagating in empty space. This can be implemented in the
oscillation formalism by adding a potential term to the Hamiltonian for
electron neutrinos only. This term will be proportional to the density of
matter and to the weak coupling constant\cite{zuber}.

In what follows, we derive the corrections that must be applied to the
oscillation formalism for two neutrino flavours, in order to accurately
describe oscillations in matter.
This derivation is inspired by section~8.9 of Kai Zuber's textbook\cite{zuber}
and Stefania Ricciardi's notes on oscillations in matter\cite{ricciardi}.

We start from the time evolution of two mass eigenstates (Schrödinger's
equation):
\begin{align}
	\mi \frac{\md}{\md t} \begin{bmatrix} \nu_1 \\ \nu_2 \end{bmatrix}
		&= \hat{H}_\text{mass} \begin{bmatrix} \nu_1 \\ \nu_2
		\end{bmatrix}\nonumber \\
		&= \begin{bmatrix} E_1 & 0 \\ 0 & E_2 \end{bmatrix} \begin{bmatrix} \nu_1
\\ \nu_2 \end{bmatrix}\nonumber \\
&= \frac{1}{2E} \bmat{m^2_1 & 0 \\ 0 & m^2_2} \bmat{\nu_1 \\ \nu_2} + \bmat{E &
0 \\ 0 & E} \bmat{\nu_1 \\ \nu_2},\label{eq:massvacuum}
\end{align}
where in the last step we used the approximation of eq.~\ref{eq:binomial}.
The second term is a multiple of the identity, and only contributes a common phase to
the wavefunctions $\nu_1$ and $\nu_2$. It does not affect the oscillations, and
will be discarded in what follows.
Now we transform into the flavour basis by using $\ket{\nu_\text{mass}} =
U^\dagger \ket{\nu_\text{flavour}}$ (eq.~\ref{eq:bases2}), and multiply by $U$ on the left:
\begin{align*}
	\mi \frac{\md}{\md t}\bigg( U^\dagger \bmat{ \nu_e \\ \nu_\mu
	}\bigg)
		&= \frac{1}{2E}\bmat{m^2_1 & 0 \\ 0 & m^2_2} U^\dagger \bmat{\nu_e
			\\ \nu_\mu} \\
		\implies 
	\mi \underbrace{U U^\dagger}_{=1} \frac{\md}{\md t}\bmat{\nu_e \\ \nu_\mu}
		&= \underbrace{\frac{1}{2E} U \bmat{m^2_1 & 0 \\ 0 & m^2_2}
			U^\dagger }_{\equiv \hat{H}_{\text{flavour}}} \bmat{\nu_e
			\\ \nu_\mu}.
\end{align*}
For this illustration, our two flavours are $\nu_e$ and $\nu_\mu$ because matter effects play
an important role in the oscillations between those two, for example
in long baseline experiments, or for solar neutrinos.
We use the explicit form of $U$ for two neutrinos (eq.~\ref{eq:twonumixing}) to
determine $\hat{H}_\text{flavour}$. The result is
\begin{equation}\hat{H}_\text{flavour} = \frac{\Delta m^2}{4 E} \bmat{-\cos 2\theta & \sin
	2\theta \\ \sin 2\theta & \cos
2\theta},\label{eq:flavourvacuum}\end{equation}
where $\Delta m^2 = m_2^2 - m_1^2$, and we have subtracted a multiple of the
identity. Our goal, after introducing the extra potential in matter, will be to
express the flavour Hamiltonian in this form, albeit with different $\Delta
m^2$ and $\theta$, and to relate it to the mass eigenstates in matter. This
will ensure that we can simply apply the steps of section~\ref{sec:twonu} and
directly obtain the oscillation formula.

In matter, the electron neutrinos interact with electrons, so we must add a
potential term $V_e$ to the Hamiltonian. For simplicity, we assume that this
term is constant, i.e. that the density of matter is constant. This assumption
is fairly accurate for neutrinos propagating in the Earth's crust, but not for
solar neutrinos, for example. The Hamiltonian in matter is
\begin{equation}
	\hat{H}^m_\text{flavour} = \frac{\Delta m^2}{4 E} \bmat{-\cos 2\theta & \sin
	2\theta \\ \sin 2\theta & \cos 2\theta} + \bmat{V_e & 0 \\ 0 &
	0}.\label{eq:matter_potential}
\end{equation}
Again, we subtract a multiple of the identity $V_e / 2$ without affecting the
oscillations:
\begin{equation}\hat{H}^m_\text{flavour} = \frac{\Delta m^2}{4 E} \bmat{-\cos 2\theta + A & \sin
2\theta \\ \sin 2\theta & \cos 2\theta - A},\label{eq:flavourmatter1}\end{equation}
where $A = 2 E V_e / \Delta m^2$.
This form of the Hamiltonian implies that the mass eigenstates in the vacuum
are not mass eigenstates in matter. Indeed, if we go back to the mass
eigenbasis, we see that the Hamiltonian is no longer diagonal:
\begin{align*}
	\hat{H}^m_\text{mass} &= U^\dagger \hat{H}^m_\text{flavour} U\\
	&= \frac{\Delta m^2}{4E} \bmat{A \cos 2\theta - 1 & A \sin 2\theta \\ A \sin
	2\theta & 1 - A\cos 2\theta}.
\end{align*}
It can be diagonalized by transforming the mass eigenvector such that
\begin{align}\mi \frac{\md}{\md t} \bmat{\nu_{1m} \\ \nu_{2m}} &= \underbrace{\frac{1}{2E}
\bmat{m^2_{1m} & 0 \\ 0 & m^2_{2m}}}_{\equiv \hat{H}^m_\text{mass, diag}}
\bmat{\nu_{1m} \\ \nu_{2m}},\label{eq:massmatter}\end{align}
where $\nu_{1m}$ and $\nu_{2m}$ are the mass eigenstates in this new basis, and
the eigenvalues $m^2_{1m}$ and $m^2_{2m}$ obey
\begin{equation}\Delta m^2_m = m^2_{2m} - m^2_{1m} = \Delta m^2 \sqrt{(\cos 2\theta - A)^2 +
\sin^2 2\theta}.\label{eq:deltam2m}\end{equation}
Equation~\ref{eq:massmatter} is the equivalent of eq.~\ref{eq:massvacuum} in
matter. The matter interaction has the effect of changing the mass difference
between our mass eigenstates. This will effectively change the frequency of the
oscillations. We now try to relate the new mass eigenstates to the old flavour
eigenstates. We can always define
$$\bmat{\nu_e \\ \nu_\mu} \equiv \underbrace{\bmat{\cos \theta_m & \sin \theta_m \\ -\sin
\theta_m & \cos \theta_m}}_{U_m} \bmat{\nu_{1m} \\ \nu_{2m}}.$$
Following the same steps as equations~\ref{eq:massvacuum}
to~\ref{eq:flavourvacuum}, this definition is equivalent to redefining the
flavour Hamiltonian in matter as
\begin{equation}\hat{H}_\text{flavour} = \frac{\Delta m^2_m}{4 E} \bmat{-\cos 2\theta_m &
\sin 2\theta_m \\ \sin 2\theta_m & \cos 2\theta_m}.\label{eq:flavourmatter2}\end{equation}
Equating equations~\ref{eq:flavourmatter2} and~\ref{eq:flavourmatter1}, we
finally get the relationship between our old and new mixing angles:
\begin{align}
	\frac{\Delta m^2}{4 E} \bmat{-\cos 2\theta & \sin 2\theta \\ \sin 2\theta &
	\cos 2\theta} &\overset{!}{=} \frac{\Delta m^2_m}{4 E} \bmat{-\cos 2\theta_m &
	\sin 2\theta_m \\ \sin 2\theta_m & \cos 2\theta_m} \nonumber\\\implies
	\sin 2\theta_m &= \sin 2\theta \frac{\Delta m^2}{\Delta m^2_m}\nonumber\\
	&= \frac{\sin 2\theta}{
		\sqrt{(\cos 2\theta - A)^2 + \sin^2 2\theta}}\label{eq:thetam}%\\\implies
	%\cos 2\theta_m &= (\cos 2\theta - A) \frac{\Delta m^2}{\Delta m^2_m}\\
	%&= (\cos
	%2\theta - A) \sqrt{(\cos 2\theta - A)^2 + \sin^2 2\theta}
\end{align}
Because we diagonalized the Hamiltonian in the mass basis and wrote the flavour
Hamiltonian in this form, equation~\ref{eq:twonu} is automatically valid and
our oscillation probability is
\begin{equation}
	P^m_{\nu_e \rightarrow \nu_\mu}(L, E) = \sin^2(2\theta_m)
	\sin^2\bigg(\frac{\Delta m^2_m L}{4 E}\bigg),\label{eq:twonumatter}
\end{equation}
where the parameters in matter are given by equations~\ref{eq:deltam2m}
and~\ref{eq:thetam}.

This derivation shows that matter effects can be taken into account by a simple
redefinition of the oscillation parameters, at least in the case of two
neutrino flavours. For three neutrinos, the algebra is a lot more involved, and
beyond the scope of this report. Ohlsson and Snellman\cite{ohlsson} have
shown that the same idea applies for three neutrino flavours, namely that we can
express all oscillation parameters in matter by diagonalizing the matter
Hamiltonian, and obtain the probability through the same formula
\begin{equation}
	P^m_{\alpha \rightarrow \beta}(L, E) = \sum_i \sum_j U^m_{\alpha i}
	U^{m*}_{\beta i} U^{m*}_{\alpha j} U^m_{\beta j} \exp(-\mi \frac{\Delta
	\tilde{m}^2_{ij} L}{2 E}),
\end{equation}
where $U^m$ and $\Delta \tilde{m}^2_{ij}$ can be expressed in terms of $U$,
$\Delta m_{ij}$ and the eigenvalues and components of the Hamiltonian in
matter. Although they only considered the case where $\delta_{CP}=0$,
expressions for the parameters were later derived\cite{kneller} in the general
case, for any $\delta_{CP}$.

Experiments in which neutrinos spend a significant amount of time propagating
through matter --- e.g. neutrinos from the sun's core, or neutrinos propagating
underground through the Earth --- require us to take matter effects into
account in order to produce a realistic model. The DUNE experiment will be
performed entirely underground with a baseline of 1300km, enough to have a
significant effect on the oscillation probability from muon neutrinos to
electron neutrinos, as we will demonstrate in the next chapter. 


\subsection{The state of neutrino oscillations}\label{sec:state}
Neutrino oscillations are still far from being fully understood. Aside from the
fact that they are not compatible with the Standard Model, the phenomenology of
oscillations is not yet complete. While three of the parameters are now known
with uncertainties of a few percent, the remaining three are expected to
be fully constrained only by the current and next generation of experiments.
%Initially, solar neutrino experiments were responsible for the determination of
%$\theta_{12}$ and $\Delta m^2_{21}$, while $\theta_{23}$ and $|\Delta m^2_{32}|$
%were measured by atmospheric neutrino experiments\cite{gonzalez-garcia}. 

\subsubsection{Known parameters}
The solar mixing angle $\theta_{12}$ and the solar mass squared difference
$\Delta m^2_{21}$ were the very first parameters to be probed, since already in
1968 Davis et al.\cite{davis} reported a discrepancy between the expected
and observed neutrino fluxes from the sun. In 2001, the SNO
collaboration\cite{sno} reported direct evidence that the solar neutrino flux
consisted only partly of electron-neutrinos by comparing their results to those
of Super-K\cite{superk-solar}.
The solar neutrino problem was solved in 2002 by the KamLAND
experiment\cite{kamland}. By observing the disappearance of $\overline{\nu}_e$
produced in several nuclear reactors, they were able to exclude all solutions to the
solar neutrino problem except for the ``Large Mixing Angle'' solution, whereby
the solar parameters were estimated to be $\Delta m^2_{21} \approx
7.5\times10^{-5}$ eV$^2$ and $\theta_{12} \approx 0.59$.

\begin{wraptable}{r}[2cm]{0.36\textwidth}
	\captionsetup{justification=centering}
	\centering
\begin{tabular}{ | c | c | }
	\hline
	Parameter & Value\\\hline
	$\Delta m^2_{21}$ & \SI{7.37e-5}{\eV^{-2}}\\
	$|\Delta m^2_{32}|$ & \SI{2.56e-3}{\eV^{-2}}\\
	$\theta_{12}$		  & 0.576\\
	$\theta_{23}$			& 0.710\\
	$\theta_{13}$ 		& 0.147\\\hline
\end{tabular}
	\caption{Best-fit oscillation parameters from a global fit of past
	experiments.}
	\label{tab:threenu_params}
\end{wraptable}
In 1998, the Super-K collaboration\cite{superk} provided the first overwhelming evidence for
the flavour oscillations of neutrinos through the observation of $\nu_\mu$
disappearance in the flux of neutrinos produced by cosmic rays in the
atmosphere. The results were consistent with values for the atmospheric
parameters of $|\Delta m^2_{32}| \approx 10^{-3}$~eV$^2$ and $\sin^2
2\theta_{23} > 0.82$. These results were later improved upon by the
MINOS\cite{minos}, NOvA\cite{nova} and T2K\cite{t2k} accelerator-based experiments.

The third mixing angle, $\theta_{13}$, couples atmospheric and solar
mixings in the PMNS matrix (see eq.~\ref{eq:PMNS}). In addition, it is evident
in the PMNS parametrization that a value of $\theta_{13} = 0$ would make the
observation of CP violation in the lepton sector impossible. Hence this
parameter plays a central role in the phenomenology of neutrino oscillations.
Determining whether or not it is zero was one of the challenges of the past
decade for neutrino physics. In 2011, T2K\cite{t2k-13} rejected $\theta_{13}=0$
with a $2.5\sigma$ significance. In 2012, the Daya Bay\cite{dayabay},
RENO\cite{reno} and Double Chooz\cite{chooz} reactor experiments reported their
first results from $\overline{\nu}_e$ disappearance measurements: they
established a small but non-zero $\theta_{13}$ with $\sin^2 2\theta_{13}
\approx 0.1$, thus confirming the possibility of observing CP violation in
subsequent neutrino oscillation experiments.

Table~\ref{tab:threenu_params} lists the current best-fit values for
the oscillation parameters, obtained from global fits of past experiments,
and reported by the Particle Data Group\cite{pdg}.


\subsubsection{Unknown parameters}
While the absolute value of $\Delta m^2_{32}$ is known with high precision,
previous experiments have not been sensitive to its sign.  The problem of
whether the third mass eigenstate is heavier or lighter than the other two is
called the neutrino mass hierarchy --- or mass ordering --- problem. It is
likely to be resolved by the next generation of long baseline neutrino
experiments, as we will show in our results.  Throughout this report, we will
denote by ``normal hierarchy'' (NH) the case where the third mass eigenstate is
heavier than the other two ($m^2_3 > m^2_2 > m^2_1$, and $\Delta m^2_{32}$ is a
positive value), and by ``inverted hierarchy'' (IH) the case where it is
lighter ($m^2_2 > m^2_1 > m^2_3$, $\Delta m^2_{32}$ negative).

The second unknown is the octant of $\theta_{23}$. This parameter is known to
lie around \SI{45}{\degree}, in either the first or second octant of the unit
circle. However, experiments in the past have typically only been
sensitive to $\sin^2{2 \theta_{23}}$, resulting in two possible degenerate
values for $\theta_{23}$ that could fit the same data\footnote{Since
$\sin(2(\pi/4+x)) = \sin(2(\pi/4-x))$ for any $x$.}.
In order to break this degeneracy, a possible solution in accelerator-based
experiments is to reverse the polarity of the pion-focusing field, hence
making the beam contain anti-neutrinos rather than neutrinos. Counting
$\nu_e$ appearance events under both polarities breaks the $\sin^2 2\theta_{23}$
degeneracy\cite{raut}, enabling the measurement of the octant of $\theta_{23}$.

%design an
%experiment with \emph{more} matter effects\cite{raut}, e.g. a long baseline
%underground experiment. DUNE will be such an experiment, as well as
%T2HKK, a proposal to build a second detector in Korea for the J-PARC neutrino
%beam in Tokai, Japan, with a baseline of 1000\textasciitilde1300 km\cite{t2hkk}.

The last unknown is the CP violating phase $\delta_{CP}$. Unlike the other
parameters, there are so far few constraints on its value.
Because of its relationship to CP violation in the lepton sector, this
parameter has a cosmological relevance since it could explain in part the
overwhelming presence of matter over antimatter in the universe. Similarly to
the other two parameters, it is expected that the next generation of neutrino
experiments will determine $\delta_{CP}$.

Although we use the standard three-neutrino framework throughout this report,
it is relevant to mention that it may not be entirely accurate. The existence of
sterile neutrinos --- neutrinos which do not interact via the weak interaction,
but do mix with the others via oscillations --- has not been ruled out by
oscillation experiments. The evidence for the existence of such particles would
be the non-unitarity of the three-neutrino mixing matrix. Since
$\delta_{CP}$ is still not fully constrained, it is so far impossible to rule out
sterile neutrinos.
