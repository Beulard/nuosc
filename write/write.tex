\documentclass[10pt, a4paper]{article}
\title{Project}
\author{Matthias Dubouchet}
\date{}

\begin{document}
\maketitle
\section{Two neutrino oscillations}
In the case of two neutrino flavors, there can only be one mixing angle
(analogous to a rotation in a plane). Get formulas from ZUBER.


\section{Three neutrino oscillations}

\section{CP violating phase}

\section{Sensitivities}

\subsection{Delta chi squared}
Let's work under the assumption that the true mass hierarchy (the one chosen by
nature) is the \emph{normal} one. When we oscillate our initial spectrum over
the baseline and obtain a prediction for the far detector spectrum, we are,
under this assumption, modeling the true oscillations as they will happen in
the real experiment. Since the CP violating phase $\delta_{CP}$ is still
experimentally unknown, it is useful to perform the neutrino oscillations while
changing $\delta_{CP}$ over its allowed range $[-\pi,\pi)$. For each of these
spectra, we are thus making two assumptions: one is the true mass hierarchy,
and the other is the true CP violating phase.

Now we ask: if our assumptions are wrong and either the true mass hierarchy
is inverted, or the phase is not what we think, or both: how accurately can
we rule out our initial assumptions? To answer this, we repeat the oscillation
with inverted hierarchy parameters, and again, we explore every value of
$\delta_{CP}$. A good way to define the agreement between two sets of data (or
in this case, predictions) is
the $\chi^2$ statistic. We calculate a $\chi^2$ between our \emph{assumed true}
spectrum and each of the \emph{inverted hierarchy} spectra. The smaller the
$\chi^2$, the better the fit, hence the minimum $\chi^2$, for a given assumed
$\delta_{CP}$, represents the best possible agreement with the opposite
hierarchy. This particular statistic is called the mean $\Delta\chi^2$
($\overline{\Delta\chi^2}$), and since it tells us how ``badly'' our experiment
might differentiate between different values of a parameter,  it is a good
measure of the \emph{sensitivity} of the experiment to this underlying parameter.
We call the quantity $\sqrt{\overline{\Delta\chi^2}}$ the sensitivity.


\section{References}
Kai Zuber, Neutrino Physics

\end{document}
